% Considera��es finais
\chapter{Considera��es finais}

As considera��es finais formam a parte final (fechamento) do texto, sendo dito de forma resumida (1) o que foi desenvolvido no presente trabalho e quais os resultados do mesmo, (2) o que se p�de concluir ap�s o desenvolvimento bem como as principais contribui��es do trabalho, e (3) perspectivas para o desenvolvimento de trabalhos futuros, como listado nos exemplos de se��o abaixo. O texto referente �s considera��es finais do autor deve salientar a extens�o e os resultados da contribui��o do trabalho e os argumentos utilizados estar baseados em dados comprovados e fundamentados nos resultados e na discuss�o do texto, contendo dedu��es l�gicas correspondentes aos objetivos do trabalho, propostos inicialmente.


\section{Principais contribui��es}

Texto.


\section{Limita��es}

Texto.


\section{Trabalhos futuros}

Texto.